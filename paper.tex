\documentclass{llncs}
\usepackage{graphics}
\usepackage{amssymb}
\usepackage[dvips]{epsfig}
\usepackage[spanish]{babel}
%\usepackage[latin1]{inputenc}

\def\CC{{C\hspace{-.05em}\raisebox{.4ex}{\tiny\bf ++}}~}
\addtolength{\textfloatsep}{-0.5cm}
\addtolength{\intextsep}{-0.5cm}


%%%%%%%%%%%%%%%% Titulo %%%%%%%%%%%%%%%
\title{SIPESCA-B: a set of real data time series benchmark for traffic prediction}

%%%%%%%%%%%%%%%% autores %%%%%%%%%%%%%%%
\author {
P.A. Castillo et al.
}
\institute{Department of Architecture and Computer Technology. CITIC \\
           University of Granada (Spain) \\
~\\
           e-mail: {\tt pacv@ugr.es}}

\date{} 

\begin{document}
\maketitle

%%%%%%%%%%%%%%%%%%%%%%%%%%%%%%%%%%%%%%%%%%%%%%%%%%%%%%%%%%%%%%
\begin{abstract}

%En el �mbito de la gesti�n del tr�fico no existen benchmarks p�blicos obtenidos a partir de datos reales disponibles para la comunidad, de forma que puedan probar sus m�todos de extracci�n de informaci�n y comparar sus resultados con los obtenidos por otros investigadores. 
In the traffic management field no public real-data benchmarks are available to the research community, so that they can test their information extracting methods and compare their results with those obtained by other researchers.
%En este trabajo se presenta el benchmark SIPESCA-B, un conjunto de series temporales obtenidas a partir de la monitorizaci�n del tr�fico en varios puntos de la red de carreteras andaluza. Se trata de datos reales relativos al n�mero de veh�culos que han pasado por ciertos puntos a lo largo de varios meses. Las series de datos se facilitan en formatos sencillos y ampliamente extendidos y se han dejado disponibles en un repositorio p�blico.
In this work the benchmark SIPESCA-B, a set of time series obtained monitoring the traffic at several locations in the Andalusian road network, is presented. 
Time series are build using real data from the number of vehicles that have passed through certain locations along several months.
The datasets are provided in simple and widespread formats and are left available in a public repository.
%El objetivo de este trabajo es ofrecer acceso y detalles del benchmark propuesto a otros investigadores, as� como unos resultados preliminares con varios m�todos de predicci�n que sirvan para la comparaci�n de futuros resultados publicados en el �rea de la gesti�n y predicci�n del tr�fico.
Our aim with is providing public access and details of this benchmark as well as preliminary results with various prediction methods to the research community to serve as comparison of future results in the management and traffic prediction field.

\end{abstract}

%\begin{keywords}
  %Traffic flow forecasting, Bluetooth technology, Prediction, Time Series
  %Benchmarks, Traffic Prediction, Bluetooth, Time Series
%\end{keywords}

%********************************************************************************
\section{Introduction}



%********************************************************************************
\section{Time series data collection}



%********************************************************************************
\section{Error measures}



%********************************************************************************
\section{Time series prediction methods}



%********************************************************************************
\section{Obtained results}



%********************************************************************************
\section{Conclusions and future work}



%********************************************************************************
\section*{Acknowledgements}
This work has been supported in part by SIPESCA (Programa Operativo FEDER de Andaluc�a 2007-2013), TIN2011-28627-C04-02 and TIN2014-56494-C4-3-P (Spanish Ministry of Economy and Competitivity), SPIP2014-01437 (Direcci{\'o}n General de Tr{\'a}fico), PRY142/14 (Fundaci{\'o}n P{\'u}blica Andaluza Centro de Estudios Andaluces en la IX Convocatoria de Proyectos de Investigaci{\'o}n), and PYR-2014-17 GENIL project (CEI-BIOTIC Granada).


%********************************************************************************
\bibliographystyle{plain}
\bibliography{refs}

\end{document}
