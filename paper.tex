\documentclass{llncs}
\usepackage{graphics}
\usepackage{amssymb}
\usepackage[dvips]{epsfig}
\usepackage[spanish]{babel}
%\usepackage[latin1]{inputenc}

\def\CC{{C\hspace{-.05em}\raisebox{.4ex}{\tiny\bf ++}}~}
\addtolength{\textfloatsep}{-0.5cm}
\addtolength{\intextsep}{-0.5cm}


%%%%%%%%%%%%%%%% Titulo %%%%%%%%%%%%%%%
\title{SIPESCA-B: a set of real data time series benchmark for traffic prediction}

%%%%%%%%%%%%%%%% autores %%%%%%%%%%%%%%%
\author {
P.A. Castillo et al.
}
\institute{Department of Architecture and Computer Technology. CITIC \\
           University of Granada (Spain) \\
~\\
           e-mail: {\tt pacv@ugr.es}}

\date{} 

\begin{document}
\maketitle

%%%%%%%%%%%%%%%%%%%%%%%%%%%%%%%%%%%%%%%%%%%%%%%%%%%%%%%%%%%%%%
\begin{abstract}

In the field of traffic management                                     % management? Prediction? Traffic what? Rate? modelling? 
no public real-data benchmarks are available to the research
community % that we know of, siempre
for testing their information processing methods and compare their
results with those obtained by other researchers.                       % This implies that... or this is bad because... - JJ
In this work the benchmark SIPESCA-B, a set of time series 
obtained monitoring the traffic at several locations in the 
Andalusian road network, is presented. 
Time series are built using real data from the number of vehicles that
have passed through certain locations along several months.             % not _all_ vehicles - JJ
The datasets are provided in simple formats with widespread use and
are left available in a public repository. 
Our aim for this work is to provide public access and details of this benchmark
as well as preliminary results with various prediction methods to the
research community to serve as comparison of future results in the field of
management and traffic prediction. 

\end{abstract}

%\begin{keywords}
  %Traffic flow forecasting, Bluetooth technology, Prediction, Time Series
  %Benchmarks, Traffic Prediction, Bluetooth, Time Series
%\end{keywords}

%********************************************************************************
\section{Introduction}

There are many fields where standard benchmarks are widely used by researchers, such as Proben1 \cite{Prechelt1994} in the artificial neural networks area, or the functional optimization competition benchmark of IEEE Congress on Evolutionary Computation \cite{CEC2014}.

However, in other areas there is a need for this type of real problems, such as in the area of traffic management.
In this area, due to the type of problems, there is a difficulty in obtaining and using real data.
In some cases due to the lack of data in a suitable format \cite{Flyvbjerg2008}, and in other cases because it is confidential commercial data \cite{Bain2009} obtained, i.e., in highway tolls \cite{Kriger2006}. 
Thus, only certain research works can use that data, which makes impossible to compare results with other researchers.

% Department of Infraestructure and Transport. Australian Government. Review the Traffic Forecasting Performance Toll Roads. http://www.infrastructure.gov.au/infrastructure/public_consultations/files/attach_a_bitre_literature_review.pdf
%
% Kriger2006
% toll data  /  data required for traffic modelling may include, among other things, traffic counts, network characteristics and generalised travel costs. These data are sometime lacking or subject to sampling/processing errors
%
%Flyvbjerg2008
% Inappropriate models, poor quality and/or lack of data, and inadequate modelling assumptions are often cited in the literature as the main sources of forecasting errors.
%
% Bain2009
%http://ibtta.org/sites/default/files/Error%20and%20optimism%20in%20traffic%20predictions.pdf
% absence of comparative data
% available toll road data 
% The author had access to commercial-in-confidence documentation [...] compiled a database of predicted and actual traffic usage for over 100 international, privately financed toll road projects
% 
% \cite{Morzy2007}
% Synthetic datasets were generated using Network-based Generator of Moving Objects by T.Brinkhoff \cite{Brinkhoff2002}. 
%
% \cite{PLAISANT2008}
% �facilitate the comparison of different techniques and encourage researchers to work on challenging problems�

That is why many researchers present their work using traffic
simulators \cite{Morzy2007} or artificial benchmarks to obtain data to
which applying data mining or prediction algorithms. % The existence
                                % of benchmarks contradicts what you
                                % said above - JJ
%
% Linear Road Benchmark simulates a variable tolling system [6]
% The system uses the MIT Trac Simulator to generate moving vehicles
%
% The DynaMark Benchmark simulates the movement of mobile users  who update their location with an average periodicity. [14]
%
% COST Benchmark  simulated objects with periodically updated 2-D locations.   [12]
%
% The BerlinMOD benchmark, like LRB, simulates spatio-temporal data of moving vehicles on a road network [9]. 
%
% HE GSMARK BENCHMARK: The resulting data consists of a real road network with simulated moving vehicles \cite{Shen2011}
%
% \cite{Gidofalvi2010}
% data set contains the GPS readings of 1500 taxis and 400 trucks travelling on the streets of Stockholm
%
Among these synthetic benchmarks we can mention:
\begin{itemize}
  \item  \emph{Linear Road Benchmark} uses the \emph{MIT Trac Simulator} to simulate the movement of vehicles on a toll \cite{Arasu2004}.
  \item  \emph{DynaMark Benchmark} simulates the movement of users, updating its position with a certain frequency \cite{Myllymaki2003}.
  \item  \emph{COST Benchmark} simulates moving objects and periodically updates its 2D position \cite{Jensen2006}.
  \item  \emph{BerlinMOD Benchmark} simulates spatio-temporal data from vehicle movements on a real road network \cite{Duntgen2009}.
  \item  \emph{GSMARK Benchmark} generates data from a real road on which simulating the vehicle movements \cite{Shen2011}.
  \item  \emph{Transport and Logistics Division of the Department of Urban Planning and Environment} public benchmark, consisting of real data from the GPS position of 1500 taxis and 400 trucks moving through the streets of Stockholm \cite{Gidofalvi2010}.
\end{itemize}


Thus there is a need for, not only standard benchmarks based on real traffic data, but also rules or conventions about how to use them to evaluate different prediction methods in the traffic management field.
One way to face this problem is to encourage researchers to either use standard benchmarks or publish not only the results but also the problem data along its detailed description.

As stated, it is not enough using a set of problems and rules; researchers should take into account that the results obtained using these problems must be comparable and reproducible.
In order to achieve this, some benchmark results are necessary as a basis for comparisons.

Having the real data based benchmark, application rules, documentation and some results for the sake of comparison, facilitates the researchers' work, ensuring reproducibility and comparability of results \cite{PLAISANT2008}.

In this sense, SIPESCA-B is intended as a first step towards a standard benchmark for the traffic management field. It consist of several time series obtained from real data from passing vehicles through several geographical points of Andalusian roads.
Making it publicly available facilitates the access to researchers to real mobility data, using simple and widespread formats.

In addition, following the recommendations of Prechelt \cite{Prechelt1994}, a set of rules are proposed to carry out the implementation and use of prediction methods to face these problems.

Finally, using real data problems versus artificial or synthetic data problems makes research and its results are relevant in at least one field \cite{Prechelt1994,PLAISANT2008}.


The rest of this paper is organized as follows:
In Section \ref{sec:timeseries}, the research project within this work has been developed is presented. This section details how the time series data has been obtained.
Section \ref{sec:medidasdeerror} proposes several standar measures of error to be used when carrying out the time series predictions, so that the obtained results are comparable with those presented by other authors.
Then, a brief state of the art on methods for predicting time series is presented (Section \ref{sec:ForecastTools}).
In Section \ref{sec:Experiments}, a series of experimental results is presented on several time series, using several prediction methods and the proposed measures of error.
Finally, Section \ref{sec:Conclusions} presents a brief conclusions, followed by future works.


%********************************************************************************
\section{Time series data collection}



%********************************************************************************
\section{Error measures}



%********************************************************************************
\section{Time series prediction methods}



%********************************************************************************
\section{Obtained results}



%********************************************************************************
\section{Conclusions and future work}



%********************************************************************************
\section*{Acknowledgements}
This work has been supported in part by SIPESCA (Programa Operativo FEDER de Andaluc�a 2007-2013), TIN2011-28627-C04-02 and TIN2014-56494-C4-3-P (Spanish Ministry of Economy and Competitivity), SPIP2014-01437 (Direcci{\'o}n General de Tr{\'a}fico), PRY142/14 (Fundaci{\'o}n P{\'u}blica Andaluza Centro de Estudios Andaluces en la IX Convocatoria de Proyectos de Investigaci{\'o}n), and PYR-2014-17 GENIL project (CEI-BIOTIC Granada).


%********************************************************************************
\bibliographystyle{plain}
\bibliography{refs}

\end{document}
